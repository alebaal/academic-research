%%%%%%%%%%%%%%%%%
% This is an sample CV template created using altacv.cls
% (v1.3, 10 May 2020) written by LianTze Lim (liantze@gmail.com). Now compiles with pdfLaTeX, XeLaTeX and LuaLaTeX.
% This fork/modified version has been made by Nicolás Omar González Passerino (nicolas.passerino@gmail.com, 15 Oct 2020)
%
%% It may be distributed and/or modified under the
%% conditions of the LaTeX Project Public License, either version 1.3
%% of this license or (at your option) any later version.
%% The latest version of this license is in
%%    http://www.latex-project.org/lppl.txt
%% and version 1.3 or later is part of all distributions of LaTeX
%% version 2003/12/01 or later.
%%%%%%%%%%%%%%%%

%% If you need to pass whatever options to xcolor
\PassOptionsToPackage{dvipsnames}{xcolor}
\usepackage{hyperref}

%% If you are using \orcid or academicons
%% icons, make sure you have the academicons
%% option here, and compile with XeLaTeX
%% or LuaLaTeX.
% \documentclass[10pt,a4paper,academicons]{altacv}

%% Use the "normalphoto" option if you want a normal photo instead of cropped to a circle
% \documentclass[10pt,a4paper,normalphoto]{altacv}

\documentclass[10pt,a4paper,ragged2e,withhyper]{altacv}

%% AltaCV uses the fontawesome5 and academicons fonts
%% and packages.
%% See http://texdoc.net/pkg/fontawesome5 and http://texdoc.net/pkg/academicons for full list of symbols. You MUST compile with XeLaTeX or LuaLaTeX if you want to use academicons.

% Change the page layout if you need to
\geometry{left=1.2cm,right=1.2cm,top=1cm,bottom=1cm,columnsep=0.75cm}

% The paracol package lets you typeset columns of text in parallel
\usepackage{paracol}

% Change the font if you want to, depending on whether
% you're using pdflatex or xelatex/lualatex
\ifxetexorluatex
  % If using xelatex or lualatex:
  \setmainfont{Roboto Slab}
  \setsansfont{Lato}
  \renewcommand{\familydefault}{\sfdefault}
\else
  % If using pdflatex:
  \usepackage[rm]{roboto}
  \usepackage[defaultsans]{lato}
  % \usepackage{sourcesanspro}
  \renewcommand{\familydefault}{\sfdefault}
\fi

% ----- LIGHT MODE -----
\definecolor{SlateGrey}{HTML}{2E2E2E}
\definecolor{LightGrey}{HTML}{666666}
\definecolor{PrimaryColor}{HTML}{001F5A}
\definecolor{SecondaryColor}{HTML}{0039AC}
\definecolor{ThirdColor}{HTML}{F3890B}
\definecolor{BackgroundColor}{HTML}{E2E2E2}
\colorlet{name}{PrimaryColor}
\colorlet{tagline}{PrimaryColor}
\colorlet{heading}{PrimaryColor}
\colorlet{headingrule}{ThirdColor}
\colorlet{subheading}{SecondaryColor}
\colorlet{accent}{SecondaryColor}
\colorlet{emphasis}{SlateGrey}
\colorlet{body}{LightGrey}
\pagecolor{BackgroundColor}   
% ----- DARK MODE -----
%\definecolor{BackgroundColor}{HTML}{242424}
%\definecolor{SlateGrey}{HTML}{6F6F6F}
%\definecolor{LightGrey}{HTML}{ABABAB}
%\definecolor{PrimaryColor}{HTML}{3F7FFF}
%\colorlet{name}{PrimaryColor}
%\colorlet{tagline}{PrimaryColor}
%\colorlet{heading}{PrimaryColor}
%\colorlet{headingrule}{PrimaryColor}
%\colorlet{subheading}{PrimaryColor}
%\colorlet{accent}{PrimaryColor}
%\colorlet{emphasis}{LightGrey}
%\colorlet{body}{LightGrey}
%\pagecolor{BackgroundColor}

% Change some fonts, if necessary
\renewcommand{\namefont}{\Huge\rmfamily\bfseries}
\renewcommand{\personalinfofont}{\small\bfseries}
\renewcommand{\cvsectionfont}{\LARGE\rmfamily\bfseries}
\renewcommand{\cvsubsectionfont}{\large\bfseries}

% Change the bullets for itemize and rating marker
% for \cvskill if you want to
\renewcommand{\itemmarker}{{\small\textbullet}}
\renewcommand{\ratingmarker}{\faCircle}

%% sample.bib contains your publications
%% \addbibresource{sample.bib}

\begin{document}
    \name{Alexandre B. de Almeida}
    \tagline{Physicist}
    %% You can add multiple photos on the left or right
    \photoL{4cm}{foto285}
    
    \personalinfo{
        \email{alexandre.almeida@alumni.usp.br}\smallskip
        \phone{+55 19 991984533}
        \location{Sao Joao da Boa Vista-SP, Brazil}\\
        \linkedin{alexandre-b-almeida-37aa7212a}
        \github{alebaal}
        %\dev{johnDoe}
        %\homepage{nicolasomar.me}
        %\medium{nicolasomar}
        %% You MUST add the academicons option to \documentclass, then compile with LuaLaTeX or XeLaTeX, if you want to use \orcid or other academicons commands.
        % \orcid{0000-0000-0000-0000}
        %% You can add your own arbtrary detail with
        %% \printinfo{symbol}{detail}[optional hyperlink prefix]
        % \printinfo{\faPaw}{Hey ho!}[https://example.com/]
        %% Or you can declare your own field with
        %% \NewInfoFiled{fieldname}{symbol}[optional hyperlink prefix] and use it:
        % \NewInfoField{gitlab}{\faGitlab}[https://gitlab.com/]
        % \gitlab{your_id}
    }
    
    \makecvheader
    %% Depending on your tastes, you may want to make fonts of itemize environments slightly smaller
    % \AtBeginEnvironment{itemize}{\small}
    
    %% Set the left/right column width ratio to 6:4.
    \columnratio{0.3}

    % Start a 2-column paracol. Both the left and right columns will automatically
    % break across pages if things get too long.
    \begin{paracol}{2}
        % ----- STRENGTHS -----
        \cvsection{Skills}
            {\bf Hard}\\
            \vspace{0.2cm}
            \cvtag{Linux}
            \cvtag{Windows}
            \cvtag{Matlab}
            \cvtag{Mathematica}
            \cvtag{Visual Studio}
            \cvtag{PyCharm}
            \cvtag{Jupyter}
            \cvtag{Vim}
            \cvtag{Git}           
            \cvtag{C++}
            \cvtag{C\#}
            \cvtag{Bash}
            \cvtag{Batch}
            \cvtag{\LaTeX}
            \cvtag{Python}
            \cvtag{TensorFlow}
            \cvtag{SQL}
            \cvtag{Machine Learning}
            \cvtag{Time series}
            \cvtag{Scientific writing}
            \medskip
            
            {\bf Soft}\\
            \vspace{0.2cm}
            \cvtag{Communication}\\
            \cvtag{Creativity}
            \cvtag{Critical thinking}\\
            \cvtag{Thinking outside the box}\\
            \cvtag{Desire to learn}
            \cvtag{Honesty}\\
            \cvtag{Respectability}
            \cvtag{Cooperation}\\
            \cvtag{Resilience}
            \cvtag{Independent}\\
            %\cvtag{Patience}
            
            
            
            
        % ----- STRENGTHS -----
        
        % ----- LEARNING -----
        \cvsection{Certifications}
            \begin{itemize}
                \item Coursera
                \begin{itemize}
                    \item[-] Machine Learning, Deep Learning and TensorFlow (see following links (L): \href{https://coursera.org/share/50d1bfc2462e11112bc420d82f6eb622}{L1},
                    \href{https://coursera.org/share/e33debfdd7f6a2a39568d602dd7e50d3}{L2},
                    \href{https://coursera.org/share/0ac1fe62db824f63021bf5cbadc8effa}{L3},
                    \href{https://coursera.org/share/8efbea3377ef5eb01b909351d7c0f93c}{L4},
                    \href{https://coursera.org/share/d0e1636b353a300204e807db06b4f543}{L5},
                    \href{https://coursera.org/share/475e3f7e1c47e1dd90e153842b19823d}{L6} and
                    \href{https://coursera.org/share/a672fdfb0b627c56cbfad49b453e4be3}{L7})
                \end{itemize}
                \item Udemy
                \begin{itemize}
                    \item[-] 
                    \href{http://ude.my/UC-1e3f10fa-527a-4820-af92-9fdb1c4bc619}{Python 3: Deep Dive (Part 1)}
                    \item[-] Machine Learning A-Z ({\it in progress})
                \end{itemize}
                \item National Research University
                \begin{itemize}
                    \item[-] \href{https://coursera.org/share/88323987bae341a22c72dde06a326c6b}{Introduction to Deep Learning}
                \end{itemize}
            \end{itemize}
                    
                    
                
        % ----- LEARNING -----
        
        % ----- LANGUAGES -----
        \cvsection{Languages}
            \cvlang{Portuguese}{Native}\\
            \cvlang{English}{Advanced}
            %% Yeah I didn't spend too much time making all the
            %% spacing consistent... sorry. Use \smallskip, \medskip,
            %% \bigskip, \vpsace etc to make ajustments.
            \smallskip
        % ----- LANGUAGES -----
            
        % ----- REFERENCES -----
   %     \cvsection{References}
%            \cvref{Ref 1}{ref-1}
%            \divider
%            
%            \cvref{Ref 2}{ref-2}
%            \divider
            
%            \cvref{Ref 3}{ref-3}
%            \smallskip
        % ----- REFERENCES -----
        
        % ----- MOST PROUD -----
        %\cvsection{Most Proud of}
        %\cvachievement{\faTrophy}{Fantastic Achievement}{and some details about it}\\
        %\divider
        %\cvachievement{\faHeartbeat}{Another achievement}{more details about it of course}\\
        % \divider
        % \cvachievement{\faHeartbeat}{Another achievement}{more details about it of course}
        % ----- MOST PROUD -----
        
        %\cvsection{A Day of My Life}
        
        %Adapted from @Jake's answer from http://tex.stackexchange.com/a/82729/226
        %\wheelchart{outer radius}{inner radius}{
        %comma-separated list of value/text width/color/detail}
        % \wheelchart{1.5cm}{0.5cm}{%
        %   6/8em/accent!30/{Sleep,\\beautiful sleep},
        %   3/8em/accent!40/Hopeful novelist by night,
        %   8/8em/accent!60/Daytime job,
        %   2/10em/accent/Sports and relaxation,
        %   5/6em/accent!20/Spending time with family
        % }
        
        % use ONLY \newpage if you want to force a page break for
        % ONLY the current column
        \newpage
        
        %% Switch to the right column. This will now automatically move to the second
        %% page if the content is too long.
        \switchcolumn
        
        % ----- ABOUT ME -----
        \cvsection{About me}
            \begin{quote}
                Passionate for science and interdisciplinary problems, I have been working in the financial market since 2018 developing quantitative trading strategies for Brazilian and US stock markets. I have a solid academic background with publications in prestigious scientific journals, and one year of experience abroad at Yeshiva University, New York--NY, EUA, performing research in physics field.
            \end{quote}
        % ----- ABOUT ME -----
        
        % ----- EXPERIENCE -----
        \cvsection{Experience}
            \cvevent{Quantitative researcher}{| Saires Capital}{March 2018 -- Present}{Sao Paulo-SP, Brazil}
            \begin{itemize}
                \item I work on the development of equity trading strategies for Brazilian and US stock market. The strategies are developed by using machine learning algorithms, time series analysis and fundamentalist data. I care about to avoid look-ahead and survivorship bias. The main programming languages I use are Python and C\#. I create automated reports to follow up the strategies in papertrade and in production.
            \end{itemize}
            %\divider
                
            \cvevent{Postdoctoral Scientist}{| University of Sao Paulo}{2017 -- 2019}{Sao Paulo-SP, Brazil}
            %\divider
                
            \cvevent{Text reviewer}{| SOMOS Educação}{2018}{Sao Paulo-SP, Brazil}
        
        % ----- EDUCATION -----
        \cvsection{Eduation}
            \cvevent{Ph.D., Physics}{| University of Sao Paulo}{2012 -- 2017}{Sao Paulo, Brazil}
            \begin{itemize}
                % \item Thesis: Capillary drops and bridges at the nanoscale (CAPES scholarship)
                \item Scientific internship at {\it Yeshiva University}, New York -- NY, EUA
            \end{itemize}
            %\divider
            
            \cvevent{Master's degree, Physics}{| University of Sao Paulo}{2010 -- 2012}{Sao Paulo, Brazil}
            \begin{itemize}
                % \item Dissertation: Analysis and thermodynamic modeling of liquid bridge ruptures: Application to lung fluids (FAPESP scholarship)
            \end{itemize}
            %\divider
            
            \cvevent{Bachelor's degree, Physics}{| University of Sao Paulo}{2005 -- 2010}{Sao Paulo, Brazil}
        % ----- EDUCATION -----
        
        % ----- PROJECTS -----
        \cvsection{Publications}
            \begin{itemize}
                \item J. Phys. Chem. C 125, 5335–-5348 (2021)
                \item J. Phys. Chem. C 122, 1556–-1569 (2018)
                \item J. Phys. Chem. C 120, 1597–-1608 (2016)
                \item Physica A 392, 3409–-3416 ({2013})

            \end{itemize}
%            \cvevent{Artigos }{1}

        % ----- PROJECTS -----
    \end{paracol}
\end{document}
