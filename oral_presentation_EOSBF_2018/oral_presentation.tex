\documentclass[8pt]{beamer} 
\usetheme{Darmstadt}



\usepackage[absolute,overlay]{textpos} 		
\usepackage{caption}
\usepackage{pifont}
\usepackage[utf8]{inputenc}
\usepackage{pxfonts}
\usepackage{txfonts}
\usepackage{color}
\usepackage{xcolor}
\usepackage{colortbl}
\usepackage{graphicx}
\usepackage{rotating}
\usepackage{mwe,lmodern}
\usepackage{multimedia}
\usepackage{multirow}

\definecolor{LABM2green}{RGB}{101,135,47} 
\definecolor{LABM2greenDark}{RGB}{76,103,34} 
\definecolor{LABM2greenItem}{RGB}{134,177,66} 


\setbeamercolor{palette quaternary}{bg=black,fg=white} %section
\setbeamercolor{subsection in head/foot}{bg=LABM2greenDark,fg=white} %subsection
\setbeamercolor{palette primary}{bg=LABM2green,fg=white}  %frametitle/
\setbeamercolor{structure}{fg=LABM2greenItem} % itemize, enumerate, etc


\setbeamercolor{palette secondary}{bg=LABM2green,fg=white}
\setbeamercolor{palette tertiary}{bg=LABM2green,fg=white}
\setbeamercolor{section in toc}{fg=LABM2greenItem} % TOC sections



%\usepackage{flashmovie} %video
\setbeamertemplate{footline}[frame number] %numerar slides
\beamertemplatenavigationsymbolsempty % desativar botoes de navegacao
\newcommand{\angstrom}{\mbox{\normalfont\AA}}
\usepackage{appendixnumberbeamer}


\let\ctg\centering
\title[]{Validation of Capillarity Theory at the Nanometer Scale. II: Stability and Rupture of Water Capillary Bridges in Contact with Hydrophobic and Hydrophilic Surfaces}
\author{\\Dr. Alexandre B. de Almeida$^1$\\
          (Postdoc)\\
%       \vspace{0.1cm}					
       Prof. Dr. Nicolas Giovambattista$^2$\\ 
%       \vspace{0.1cm}				     
       Prof. Dr. Sergey V. Buldyrev$^3$\\  
%       \vspace{0.1cm}				     
       Prof. Dr. Adriano M. Alencar$^1$\\ 
%       \vspace{0.1cm}					
 	   } 
\institute{
\vspace{-0.5cm}
	\begin{center}
		$^1$ Laboratório de Microrreologia e Fisiologia Molecular (LabM$^2$) - Instituto de Física - USP - SP - Brasil \\
		$^2$ Department of Physics - Brooklyn College of the City University of New York\\ 
		$^3$ Department of Physics - Yeshiva University\\
	\end{center} 

    \begin{minipage}[c]{0.31\textwidth}
    	\begin{center}
        	\includegraphics[width=0.4\linewidth]{labm2Vertical2.png}
    	\end{center}
 	\end{minipage}
    \begin{minipage}{0.31\textwidth}
    	\begin{center}
        	\includegraphics[width=0.55\linewidth]{ifusp-logo-bw.pdf}
    	\end{center}    
 	\end{minipage} 		
	\begin{minipage}{0.31\textwidth}
    	\begin{center}
    	 	\includegraphics[width=0.4\linewidth]{usp_pro_cultura.pdf}
    	\end{center}	
	\end{minipage}
} 


\date{2018}

\graphicspath{{./figuras/}} 

\begin{document}
\frame{\titlepage}

\subsection{}



\section{Introduction}
\subsection{Capillarity at the nanoscale}

\addtobeamertemplate{frametitle}{}{%
\begin{textblock*}{100mm}(12.12cm,8.5cm)
\includegraphics[width=0.06\linewidth]{labm2Vertical2.png}
\end{textblock*}}

\begin{frame}
\frametitle{Capillarity occurs due to the \textbf{cohesion and adhesion forces} between the surface of a liquid and another medium, and it is evidenced when this liquid is subjected to the confinement.}
	\begin{minipage}{0.47\textwidth}
		\begin{block}{Capillary bridges in lung physiology}
		    \begin{center}
		    	\includegraphics[width=0.5\linewidth]{lungbridge.png}\\
		    	\tiny{[Alencar, Phys. Rev. Lett. (2001)]}
	    	\end{center} 
			\begin{itemize}
			    \item The capillary bridges can block the small airways in a diseased lung.
			    \item The bridge rupture generates the crackle sounds, which can be used as a qualitative diagnostic tool.
			\end{itemize}	    	
		\end{block}
	\end{minipage}
	~~~~~
	\begin{minipage}{0.47\textwidth}
		\begin{block}{Capillary bridges at the nanoscale}
			\begin{itemize}
			    \item Ink transport in the Dip-pen nanolithography
			    \item The capillary adhesion force causes instability and damage on the substrate when making images with AFM.
			\end{itemize}
			\begin{minipage}{0.48\textwidth}
				\begin{center}
					\includegraphics[width=0.9\linewidth]{ponte_MFA.pdf}\\
		    		\tiny{[Weeks, Langmuir ($2005$)]}			
				\end{center}		
			\end{minipage}
		    \begin{minipage}{0.48\textwidth}
				\begin{center}
					\includegraphics[width=0.9\linewidth]{Urtizberea2015.pdf}\\
					\tiny{[Urtizberea, Nanoscale ($2015$)]}		
				\end{center}		
			\end{minipage}			
		\end{block}
	\end{minipage}	
\end{frame}

\subsection{On this work}
\begin{frame}
\frametitle{Evaluate the macroscopic capillary theory (CT) at the nanoscale by studying the stability of water axisymmetric capillary bridges (AS bridges).}
   		\begin{minipage}{0.48\textwidth}
   			\begin{itemize}
   			    \item \small \textbf{Phase diagram:} dependence of height and contact angle on the rupture of capillary bridges.
   			\end{itemize}
		\begin{center}
	          \includegraphics[width=0.85\linewidth]{exemplosEtapa3.pdf}   			
		\end{center}
		\vspace{0.75cm}
   		\end{minipage}   		   				
   		~
   		\begin{minipage}{0.48\textwidth}
        {\small{\textbf{Molecular dynamics simulation} of water AS bridges attached to silicon dioxide walls.}}
			\begin{center}
				\includegraphics[width=0.7\linewidth]{fig8new.png}\\
			\end{center}
   		\end{minipage}   		
        \vspace{0.5cm}   		
		\begin{center}
			\includegraphics[width=0.7\linewidth]{coverPageAlmeida2018.png}
		\end{center}
   		
\end{frame}


\section{Methodology}
\subsection{Capillary theory}


\begin{frame}
\frametitle{Theoretical profiles $r(z)$ of stable AS bridges, and its main properties. }
    \vspace{-0.2cm}

		\begin{minipage}{0.45\textwidth}   
				\begin{center}
				 	\includegraphics[width=0.9\linewidth]{AS-bridge.pdf}\\
			 		$$\frac{{\rm d}z}{{\rm d}r} = \pm \frac{\left | H(r^{\rm 2} - r_{\rm 0}^{\rm 2}) + r_{\rm 0} \right |}{\sqrt{r^{\rm 2} -[H(r^{\rm 2} - r_{\rm 0}^{\rm 2}) + r_{\rm 0}]^{\rm 2}}}$$					 	
			 		$$r_{\rm 0} = R_{\rm 1} $$ 
			 		$$H = \frac{1}{2}  \left( \frac{1}{R_{\rm 1}} + \frac{1}{R_{\rm 2}} \right)  $$
				\end{center}
		\end{minipage}
		\begin{minipage}{0.5\textwidth}
  	   		\begin{block}<2->{Contact angle}
                   $$\theta_{\rm c}\rightarrow \frac{{\rm d}z}{{\rm d}r} \Big|_{r_{\rm B}}$$
	  	   	\end{block}	
		    \vspace{-0.3cm}	  	   	
	  	   	\begin{block}<2->{Adhesion forces and surface tension}
	  	   		   $$F = F_{\rm \gamma} +  F_{\rm P}$$
	   	           $$F_{\rm{z,base}}= 2\pi\gamma r_{\rm B}\sin\theta_{\rm c} -2\pi \gamma H r_{\rm B}^{\rm 2}$$
	   	           $$F_{\rm{z,neck}}=2\pi\gamma \mathcal C = 2\pi\gamma(r_{\rm 0} - Hr_{\rm 0}^{\rm 2})$$
                   $$F_{\rm{z,base}}=F_{\rm z,neck}$$					
	  	   	\end{block}	
		    \vspace{-0.3cm}	  	   		  	   	
	  	   	\begin{block}<2->{Laplace pressure}
		           $$P_{\rm L} =  \frac{F_{\rm \gamma} -\left\langle  F_{\rm{z,base}} \right\rangle }{A_{\rm base}}$$ 
		           $$P_{\rm L} =  2\gamma H$$
	  	   	\end{block}		  	   	
		    \vspace{-0.3cm}	  	   		  	   	
	  	   	\begin{block}<2->{Surface free energy}
			       $$\mathcal F(h) = \gamma (A_{\rm LG}(h) - \cos\theta_{\rm c} A_{\rm LS}(h))$$
	  	   	\end{block}		  	   		
		\end{minipage}		
\end{frame}		

\begin{frame}
\frametitle{Analytical solution of AS bridge}
   		\begin{center}
       			\includegraphics[width=0.77\linewidth]{diagraFase.pdf}\\   		
   		\end{center}   		
   			\begin{itemize}
   				\item $h_{\rm max, s}$ ($h_{\rm max, a}$) is the maximum height for symmetrical (asymmetrical) rupture.
   				\item For $h_{\rm max, s} < h_{\rm max, a}$, we can analytically calculate the $F_{\rm{z,base}}$, $\mathcal F(h)$ and $P_{\rm L}$ from the parameters $h$, $\theta_{\rm c}$ and volume $\Omega$. 
   			\end{itemize}   				
		\rule{\textwidth}{0.5pt}
        A. M. Alencar, E. Wolfe and Sergey V. Buldyrev, Phys. Rev. E \textbf{74}, 026311 (2006)\\   			
\end{frame}		

\subsection{Simulation}

\begin{frame}
	\frametitle{Molecular dynamics and atomistic model.}
	    \vspace{-0.25cm}
		\begin{block}{Silicon dioxide walls, water model and simulation parameters}
			    \vspace{-0.2cm}
				\begin{minipage}{0.23\textwidth}
						\includegraphics[width=1.15\linewidth]{schemeWall.pdf}
				\end{minipage}
				\begin{minipage}{0.34\textwidth}
   				 	\begin{itemize}
   				 		\item \small Four layers of SiO$_2$
   				 		\item \small Partial charges on silanol group  
   				     	\item \small Si and O fixed
   				     	\item \small H can move in a circle
						\item \small Changing $\theta_{\rm c}$:\\
						~~~~~~~~$\overrightarrow{p} = k\times \overrightarrow{p}_{0}$\\
						~~~~~~~~$\overrightarrow{p}_0 = \overrightarrow{p}_{SiO} + \overrightarrow{p}_{OH}$\\   				     	
   				 	\end{itemize} 	
				 \end{minipage}
				 \hspace{-0.1cm}
         		\begin{minipage}{0.41\textwidth}
                    \uncover<2->{
         			\begin{center}
	  					\includegraphics[width=0.26\linewidth]{water.pdf}\\
   					 	    \small SPC/E water model
         			\end{center}
         			}
     		        \vspace{-0.2cm} 
			    	\begin{minipage}{0.4\textwidth}
         		        \uncover<2->{
						\begin{center}
					    	\includegraphics[width=1.1\linewidth]{LAMMPS.png}\\
					    	\includegraphics[width=0.8\linewidth]{bridge.png}
						\end{center}
						}
				    \end{minipage}
                    \hspace{-0.25cm}				    
					\begin{minipage}{0.58\textwidth}
         		        \uncover<2->{
						\begin{itemize}
							\item \small $r_{\rm c}=10$\,\AA
							\item \small PPPM: $10^{-5}$
							\item \small \textit{velocity}-Verlet
							\item \small Nos\'{e}-Hoover
							\item \small SHAKE
						\end{itemize}		
						}
					\end{minipage}
      		   \end{minipage}			 
	    \vspace{-0.1cm}	   
		\end{block}
		\vspace{-0.1cm}
  	   \begin{minipage}{0.46\textwidth}
  	   	    \vspace{-0.2cm}	   
  	   		\begin{block}<3->{Simulations procedure:}
				\begin{itemize}
					\item<3-> \small Total water molecules: $3375$
					\item<3-> \small Polarity: $0<k<0.67$
  				\end{itemize}	
			    \vspace{-0.1cm}	   
  				\uncover<4->{\textbf{\small Stable heights ($h_{\rm S}$)}}
				\begin{itemize}	
					\item<4-> \small $h_f=h_i+5$\,\AA~ ($0.01\hat{k}$\,\AA for $250$\,fs)
					\item<4-> \small $1$\,ns to equilibrate  					
					\item<4-> \small $2$\,ns for the averages					
  				\end{itemize}										
  				\uncover<5->{\small \textbf{Critical heights ($h_{\rm C'}$ and $h_{\rm C}$)}}
				\begin{itemize}	
					\item<5-> \small First rupture: $h_{\rm C{}'} = h_{\rm S} + 5$\,\AA
          			\item<5-> \small Second rupture: $h_{\rm C}= h_S + 2.5$\,\AA
                    \item<5-> \small $\Delta t_{\rm sim}=20$\,ns:  $h = (h_{S} +2.5) + 1.25$\,\AA
  				\end{itemize}	  	   		      
		    \end{block}  	   		
	   \end{minipage}		
	   ~~
  	   \begin{minipage}{0.505\textwidth}
  	   		\begin{block}<6->{Calculation of the average profile $r_{\rm p}(z)$}
	    	\includegraphics[width=1.02\linewidth]{Ajuste.pdf}	    	
			\begin{itemize}	
				\item \small \hspace{-0.2cm} $r_{\rm 0}$, $r_{\rm B}$, $\theta_{\rm c}$, $H$, $\Omega$,\hspace{-0.025cm} $A_{\rm R}$ and $A_{\rm B}$
	        \end{itemize}	  	   		
			\end{block} 
	   \end{minipage}			   			 
\end{frame}



\section{Results}

\subsection{Stable AS bridges}


\begin{frame}
\frametitle{\textbf{Numerical fitting of the profiles $\mathbf{r(z)}$} (---) to the average profiles $r_{\rm p}(z)$ ($\circ$) obtained from the $2$\,ns simulation.}
 		\hspace{-1.cm}
	 	\begin{center}
        	\includegraphics[width=1\columnwidth]{./profiles.pdf}   
		\end{center}
        \begin{center}
        \resizebox{0.9\textwidth}{!}{ %	
		\begin{tabular}{ccccccccccc}
              & \multicolumn{10}{c}{{\bf Polarity (k)}}\\
              &       $0$ & $0.1$   & $0.2$   & $0.3$  & $0.4$  & $0.5$  & $0.6$  & $0.65$ & $0.66$ & $0.67$ \\ 
              \cline{2-11}
              $\theta_{\rm c}$[$^{\rm o}$] &   $105.6$ & $104.8$ & $100.8$ & $94.3$ & $84.4$ & $70.8$ & $50.4$ & $32.8$ & $28.3$ & $25.8$ \\ 
		\end{tabular} 
		} %
	\end{center}

\end{frame}

\begin{frame}
\frametitle{Contact angle as a function of (a) height and  (b) polarity $k$.}
  	  \begin{center}
		\includegraphics[width=0.95\columnwidth]{./fig3.pdf}
	  \end{center}
	  \vspace{0.5cm}
      \begin{minipage}{0.2\textwidth}
		\begin{center}
			\includegraphics[width=0.6\linewidth]{silanol.pdf}\\
     	    {\small \bf Silanol group}
		\end{center}
      \end{minipage}				
      \begin{minipage}{0.33\textwidth}
	 	 $$\overrightarrow{p} = k\times \overrightarrow{p}_{0}$$ 
	     $$\overrightarrow{p}_0 = \overrightarrow{p}_{SiO} + \overrightarrow{p}_{OH}$$ 	
      \end{minipage}		
      \begin{minipage}{0.45\textwidth}
      	\textbf{Wetting criteria:}
 	     \begin{itemize}
            \item $k > 0.35$: Hydrophilic surface;
	        \item $k < 0.35$: Hydrophobic surface.
	     \end{itemize}		
      \end{minipage}

\vspace{0.75cm}
\rule{\textwidth}{0.5pt}
\footnotesize{Ref.: N. Giovambattista, A. B. Almeida, A. M. Alencar and S. V. Buldyrev, \textit{J. Phys. Chem. C } \textbf{120}, 1597 (2016)}
\end{frame}

\begin{frame}
	\frametitle{Capillary adhesion force = Surface tension force + Laplace force.}
	\begin{center}
		\textbf{The dashed lines are the analytical solution based on $\theta$, $\Omega$ and $h$. }
	\end{center}	
	\begin{center}
		\includegraphics[width=0.95\columnwidth]{./fig5.pdf}   
	\end{center}
\end{frame}

\begin{frame}
	\frametitle{Laplace pressure and surface free energy.}
	\begin{center}
		\includegraphics[width=1.025\columnwidth]{./FreeEnergyAndPressure.pdf}   
	\end{center}
\end{frame}


\subsection{Instability of AS bridges}

\begin{frame}
\frametitle{Snapshots of water configurations at the rupture time $\tau_{\rm R}$ and at the critical height $h_{\rm C'}=h_{\rm S}+5$\,\AA.}
	\begin{center}
  	   \includegraphics<1>[width=1.0\columnwidth]{./AllSnap1.pdf}   
%  	   \includegraphics<2->[width=1.0\columnwidth]{./AllSnap1_2.pdf}
	\end{center}
	\begin{center}
      \resizebox{0.9\textwidth}{!}{ %	
		\begin{tabular}{ccccccccccc}
					             & \multicolumn{10}{c}{{\bf Polarity (k)}}\\
                                 &       $0$ & $0.1$   & $0.2$   & $0.3$  & $0.4$  & $0.5$  & $0.6$  & $0.65$ & $0.66$ & $0.67$ \\ 
    \cline{2-11}
    $\theta_{\rm c}$[$^{\rm o}$] &   $105.6$ & $104.8$ & $100.8$ & $94.3$ & $84.4$ & $70.8$ & $50.4$ & $32.8$ & $28.3$ & $25.8$ \\ 
    $\tau_{\rm R}$ (ps)          &    $1234$ & $1332$  & $881$   & $2094$ & $2514$ & $1402$ & $262$  & $334$  & $3341$ & $2159$ \\ 
		\end{tabular} 
		} %
	\end{center}
\end{frame}

\begin{frame}
\frametitle{Phase diagram ($\theta_{\rm c},h$) of AS bridges rupture.}
      \begin{minipage}{0.5\textwidth}
          	\includegraphics[width=1\columnwidth]{./TOC_Graphic.pdf}      	
      \end{minipage}	
      ~~			
      \begin{minipage}{0.45\textwidth}
      	  	\includegraphics[width=1\columnwidth]{./fig9.pdf}
      \end{minipage}

      \begin{minipage}{0.43\textwidth}
           	\includegraphics[width=1\columnwidth]{./PRE2006pd.pdf}
      \end{minipage}	
      \begin{minipage}{0.52\textwidth}
            $$\frac{\Delta N_{\rm d}}{N} = \left| \frac{N_{\rm droplet 1} - N_{\rm droplet 2}}{3375} \right|  $$ 
			\begin{itemize}
				\item {\color{red}$\times$}$\rightarrow$ First rupture: $h_{\rm C{}'} = h_{\rm S} + 5$\,\AA
				\vspace{0.2cm}
				\item {\color{black}$\times$}$\rightarrow$ Second rupture: $h_{\rm C}= h_S + 2.5$\,\AA \\
				 $\Delta t_{\rm sim}=20$\,ns: $h = (h_{S} +2.5) + 1.25$\,\AA      
			\end{itemize}
      \end{minipage}
      

\end{frame}

\section{Conclusions}
\begin{frame}
\frametitle{Conclusions}
	\vspace{-0.3cm}
	\begin{block}<1->{Stable AS bridges}
 		\begin{itemize}
			\item The macroscopic CT predicts the profile of nanometric capillary bridges.
			\item The contact angle is independent of the height.
			\item The surface tension calculation based on CT concepts is in agreement with literature. 
			\item The analytical solution ($\Omega$, $\theta_{\rm c}$ and $h$) of AS bridges predicts the forces and pressures at the nanoscale.			
		\end{itemize}
	\end{block}
	\begin{block}<2->{Unstable AS bridges}
 		\begin{itemize}
			\item The rupture heights $h_{\rm C'}$ and $h_{\rm C}$ are in agreement with analytical solution $h_{\rm max,a}$ and $h_{\rm max,s}$.			
%			\item We observed fluctuations on the profile of AS bridges which are not described by macroscopic CT, and it is subject of our future work.
		\end{itemize}
	\end{block}	
 	\begin{block}<3->{Main conclusion}
	\textbf{The macroscopic capillary theory is able to predict the properties of capillary bridges with volumes in the order of  $\mathbf{100}$\,nm$\boldsymbol{^3}$.}
 	\end{block}

\end{frame}



\begin{frame}
	\frametitle{Acknowledgements}
	\vspace{-0.13cm}
    \begin{minipage}{0.42\textwidth}
		\begin{center}
   			\includegraphics[width=0.99\linewidth]{labm2grupo.png}
    	\end{center}	   		
    \end{minipage}    
	\begin{minipage}{0.57\textwidth}
		\begin{minipage}{0.49\textwidth}
		  	\begin{center}
		   		\includegraphics[width=0.9\linewidth]{agradecimentos.png}    
		   	\end{center}	    	
		\end{minipage}    
		\begin{minipage}{0.49\textwidth}
		   	\begin{center}
		   		\includegraphics[width=0.6\linewidth]{INCTlogo.png}\\
		   		\textbf{\LARGE INCT-Fcx}
		   	\end{center}	       	
		\end{minipage}
		\begin{center}
			\includegraphics[width=0.6\linewidth]{fapesp.pdf}    
		\end{center}		            
	\end{minipage}

    \begin{minipage}{0.44\textwidth}
       \begin{itemize}
	    	\item Prof. Dr. Adriano M. Alencar
			\item Prof. Dr. Sergey V. Buldyrev
			\item Prof. Dr. Nicolas Giovambattista				
	   \end{itemize}							
    \end{minipage}
	\begin{minipage}{0.175\textwidth}
			\begin{center}
		   	 	\includegraphics[width=0.7\linewidth]{usp_pro_cultura.pdf}
    		\end{center}		   	 	
	\end{minipage}
	\begin{minipage}{0.175\textwidth}	
			\begin{center}
				\includegraphics[width=0.85\linewidth]{Yeshiva_University.pdf}
    		\end{center}				
	\end{minipage}
	\begin{minipage}{0.175\textwidth}	
			\begin{center}
				\includegraphics[width=1\linewidth]{brooklynCollege.pdf}
    		\end{center}				
	\end{minipage}
	\vspace{-0.1cm}
	\begin{block}{Main references}
		\begin{itemize}
			\item \footnotesize{A. B. Almeida, N. Giovambattista, S. V. Buldyrev and A. M. Alencar, J. Phys. Chem. C \textbf{122}, 1556 (2018)}
			\item \footnotesize{N. Giovambattista, A. B. Almeida, A. M. Alencar and S. V. Buldyrev, J. Phys. Chem. C \textbf{120}, 1597 (2016)}
		    \item \footnotesize{A. M. Alencar, E. Wolfe and S. V. Buldyrev, Phys. Rev. E \textbf{74}, 026311 (2006)}						
			\item \footnotesize A. B. Almeida, Gotas e pontes capilares na escala nanométrica. 2016, Thesis (Doctorate in Physics). Available from: http://www.teses.usp.br/teses/disponiveis/43/43134/tde-10072017-094009/.
		\end{itemize}
	\end{block}
	\begin{center}
	    \pause
		\textbf{\LARGE Thank you!!~~~~~~~~ Questions?!}
	\end{center}
\end{frame}

\begin{frame}
	\frametitle{Liquid-solid and liquid-gas interface areas, volume and surface free energy.}
	\begin{center}
		\textbf{The dashed lines are the analytical solution based on $\theta$, $\Omega$ and $h$. }
	\end{center}	
	\begin{center}
		\includegraphics[width=0.95\columnwidth]{./fig4.pdf}   
	\end{center}
\end{frame}

\begin{frame}
\frametitle{Validation of capillary theory for $h\leq50.0$\AA.}
	\begin{center}
		\includegraphics[width=0.49\linewidth]{ThetaXheight.pdf}\\
		\includegraphics[width=0.49\linewidth]{cosThetaXheight.pdf}
		\includegraphics[width=0.49\linewidth]{FreeEnergyTotXheight.pdf}\\		
	\end{center}
\end{frame}

\end{document}





