\begin{titlepage}

\centering
\large{Institute of Physics from University of Sao Paulo\\General Physics Departament\\Research Group on Microrheology and Physiology at Multiscales}
\vspace{\stretch{3}}

\Large{\bf Nanometric capillary bridges: the next step on energy harvesting}\\
\vspace{1cm}
\normalsize{Candidate: Alexandre Barros de Almeida}\\
\normalsize{Supervisor: Prof. Dr. Adriano M. Alencar}\\
\vspace{\stretch{1}}


\vspace{\stretch{5}}

\begin{quotation}
\noindent 
The capillarity phenomena at the macroscopic scale is well described by capillary theory (CT) that uses continuum surfaces to model the interface between two media.  
However, this approach may not be adequate for nanoscale systems, which is the focus of our research in recent years, in which we have shown that CT describes satisfactorily systems with volume $\approx100$\,nm$^3$.
Now, we will progress on this research by using molecular dynamics to simulate axisymmetric capillary bridges (AS bridge) constituted by water and adhered to two surfaces of $\beta$-cristobalite, which we can change the separation $h$ and the wettability, that is measured by the contact angle $\theta$.
Papers on the literature about capillarity at the nanoscale have reported anomalous behaviors such as liquid filling in nanochannels, oscillations in force measurements and negative pressure measurements.
Here, we will evaluate the wettability at the $\beta$-cristobalite surfaces and the fluctuations on the AS bridge interfaces varying $\theta$ and $h$.
Then, we will study the behavior of the AS bridges adhered to a dielectric material and investigate the possibility of fluctuations on the interface to generate electric current.
These studies belong to the thematic project FAPESP/SHELL 2017/11631-2.
All the results obtained will be compared with the analytical solution of the macroscopic CT and with the thermodynamic properties of the water already known.
The results obtained will have a direct impact on the understanding of water properties under confinement, and will open the possibility to use the fluctuations of the nanometric capillary bridges as a mechanism for the energy harvesting.
\end{quotation}

\end{titlepage}

